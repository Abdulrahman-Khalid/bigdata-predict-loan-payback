\documentclass[11pt]{article}

    \usepackage[breakable]{tcolorbox}
    \usepackage{parskip} % Stop auto-indenting (to mimic markdown behaviour)
    
    \usepackage{iftex}
    \ifPDFTeX
    	\usepackage[T1]{fontenc}
    	\usepackage{mathpazo}
    \else
    	\usepackage{fontspec}
    \fi

    % Basic figure setup, for now with no caption control since it's done
    % automatically by Pandoc (which extracts ![](path) syntax from Markdown).
    \usepackage{graphicx}
    % Maintain compatibility with old templates. Remove in nbconvert 6.0
    \let\Oldincludegraphics\includegraphics
    % Ensure that by default, figures have no caption (until we provide a
    % proper Figure object with a Caption API and a way to capture that
    % in the conversion process - todo).
    \usepackage{caption}
    \DeclareCaptionFormat{nocaption}{}
    \captionsetup{format=nocaption,aboveskip=0pt,belowskip=0pt}

    \usepackage{float}
    \floatplacement{figure}{H} % forces figures to be placed at the correct location
    \usepackage{xcolor} % Allow colors to be defined
    \usepackage{enumerate} % Needed for markdown enumerations to work
    \usepackage{geometry} % Used to adjust the document margins
    \usepackage{amsmath} % Equations
    \usepackage{amssymb} % Equations
    \usepackage{textcomp} % defines textquotesingle
    % Hack from http://tex.stackexchange.com/a/47451/13684:
    \AtBeginDocument{%
        \def\PYZsq{\textquotesingle}% Upright quotes in Pygmentized code
    }
    \usepackage{upquote} % Upright quotes for verbatim code
    \usepackage{eurosym} % defines \euro
    \usepackage[mathletters]{ucs} % Extended unicode (utf-8) support
    \usepackage{fancyvrb} % verbatim replacement that allows latex
    \usepackage{grffile} % extends the file name processing of package graphics 
                         % to support a larger range
    \makeatletter % fix for old versions of grffile with XeLaTeX
    \@ifpackagelater{grffile}{2019/11/01}
    {
      % Do nothing on new versions
    }
    {
      \def\Gread@@xetex#1{%
        \IfFileExists{"\Gin@base".bb}%
        {\Gread@eps{\Gin@base.bb}}%
        {\Gread@@xetex@aux#1}%
      }
    }
    \makeatother
    \usepackage[Export]{adjustbox} % Used to constrain images to a maximum size
    \adjustboxset{max size={0.9\linewidth}{0.9\paperheight}}

    % The hyperref package gives us a pdf with properly built
    % internal navigation ('pdf bookmarks' for the table of contents,
    % internal cross-reference links, web links for URLs, etc.)
    \usepackage{hyperref}
    % The default LaTeX title has an obnoxious amount of whitespace. By default,
    % titling removes some of it. It also provides customization options.
    \usepackage{titling}
    \usepackage{longtable} % longtable support required by pandoc >1.10
    \usepackage{booktabs}  % table support for pandoc > 1.12.2
    \usepackage[inline]{enumitem} % IRkernel/repr support (it uses the enumerate* environment)
    \usepackage[normalem]{ulem} % ulem is needed to support strikethroughs (\sout)
                                % normalem makes italics be italics, not underlines
    \usepackage{mathrsfs}
    

    
    % Colors for the hyperref package
    \definecolor{urlcolor}{rgb}{0,.145,.698}
    \definecolor{linkcolor}{rgb}{.71,0.21,0.01}
    \definecolor{citecolor}{rgb}{.12,.54,.11}

    % ANSI colors
    \definecolor{ansi-black}{HTML}{3E424D}
    \definecolor{ansi-black-intense}{HTML}{282C36}
    \definecolor{ansi-red}{HTML}{E75C58}
    \definecolor{ansi-red-intense}{HTML}{B22B31}
    \definecolor{ansi-green}{HTML}{00A250}
    \definecolor{ansi-green-intense}{HTML}{007427}
    \definecolor{ansi-yellow}{HTML}{DDB62B}
    \definecolor{ansi-yellow-intense}{HTML}{B27D12}
    \definecolor{ansi-blue}{HTML}{208FFB}
    \definecolor{ansi-blue-intense}{HTML}{0065CA}
    \definecolor{ansi-magenta}{HTML}{D160C4}
    \definecolor{ansi-magenta-intense}{HTML}{A03196}
    \definecolor{ansi-cyan}{HTML}{60C6C8}
    \definecolor{ansi-cyan-intense}{HTML}{258F8F}
    \definecolor{ansi-white}{HTML}{C5C1B4}
    \definecolor{ansi-white-intense}{HTML}{A1A6B2}
    \definecolor{ansi-default-inverse-fg}{HTML}{FFFFFF}
    \definecolor{ansi-default-inverse-bg}{HTML}{000000}

    % common color for the border for error outputs.
    \definecolor{outerrorbackground}{HTML}{FFDFDF}

    % commands and environments needed by pandoc snippets
    % extracted from the output of `pandoc -s`
    \providecommand{\tightlist}{%
      \setlength{\itemsep}{0pt}\setlength{\parskip}{0pt}}
    \DefineVerbatimEnvironment{Highlighting}{Verbatim}{commandchars=\\\{\}}
    % Add ',fontsize=\small' for more characters per line
    \newenvironment{Shaded}{}{}
    \newcommand{\KeywordTok}[1]{\textcolor[rgb]{0.00,0.44,0.13}{\textbf{{#1}}}}
    \newcommand{\DataTypeTok}[1]{\textcolor[rgb]{0.56,0.13,0.00}{{#1}}}
    \newcommand{\DecValTok}[1]{\textcolor[rgb]{0.25,0.63,0.44}{{#1}}}
    \newcommand{\BaseNTok}[1]{\textcolor[rgb]{0.25,0.63,0.44}{{#1}}}
    \newcommand{\FloatTok}[1]{\textcolor[rgb]{0.25,0.63,0.44}{{#1}}}
    \newcommand{\CharTok}[1]{\textcolor[rgb]{0.25,0.44,0.63}{{#1}}}
    \newcommand{\StringTok}[1]{\textcolor[rgb]{0.25,0.44,0.63}{{#1}}}
    \newcommand{\CommentTok}[1]{\textcolor[rgb]{0.38,0.63,0.69}{\textit{{#1}}}}
    \newcommand{\OtherTok}[1]{\textcolor[rgb]{0.00,0.44,0.13}{{#1}}}
    \newcommand{\AlertTok}[1]{\textcolor[rgb]{1.00,0.00,0.00}{\textbf{{#1}}}}
    \newcommand{\FunctionTok}[1]{\textcolor[rgb]{0.02,0.16,0.49}{{#1}}}
    \newcommand{\RegionMarkerTok}[1]{{#1}}
    \newcommand{\ErrorTok}[1]{\textcolor[rgb]{1.00,0.00,0.00}{\textbf{{#1}}}}
    \newcommand{\NormalTok}[1]{{#1}}
    
    % Additional commands for more recent versions of Pandoc
    \newcommand{\ConstantTok}[1]{\textcolor[rgb]{0.53,0.00,0.00}{{#1}}}
    \newcommand{\SpecialCharTok}[1]{\textcolor[rgb]{0.25,0.44,0.63}{{#1}}}
    \newcommand{\VerbatimStringTok}[1]{\textcolor[rgb]{0.25,0.44,0.63}{{#1}}}
    \newcommand{\SpecialStringTok}[1]{\textcolor[rgb]{0.73,0.40,0.53}{{#1}}}
    \newcommand{\ImportTok}[1]{{#1}}
    \newcommand{\DocumentationTok}[1]{\textcolor[rgb]{0.73,0.13,0.13}{\textit{{#1}}}}
    \newcommand{\AnnotationTok}[1]{\textcolor[rgb]{0.38,0.63,0.69}{\textbf{\textit{{#1}}}}}
    \newcommand{\CommentVarTok}[1]{\textcolor[rgb]{0.38,0.63,0.69}{\textbf{\textit{{#1}}}}}
    \newcommand{\VariableTok}[1]{\textcolor[rgb]{0.10,0.09,0.49}{{#1}}}
    \newcommand{\ControlFlowTok}[1]{\textcolor[rgb]{0.00,0.44,0.13}{\textbf{{#1}}}}
    \newcommand{\OperatorTok}[1]{\textcolor[rgb]{0.40,0.40,0.40}{{#1}}}
    \newcommand{\BuiltInTok}[1]{{#1}}
    \newcommand{\ExtensionTok}[1]{{#1}}
    \newcommand{\PreprocessorTok}[1]{\textcolor[rgb]{0.74,0.48,0.00}{{#1}}}
    \newcommand{\AttributeTok}[1]{\textcolor[rgb]{0.49,0.56,0.16}{{#1}}}
    \newcommand{\InformationTok}[1]{\textcolor[rgb]{0.38,0.63,0.69}{\textbf{\textit{{#1}}}}}
    \newcommand{\WarningTok}[1]{\textcolor[rgb]{0.38,0.63,0.69}{\textbf{\textit{{#1}}}}}
    
    
    % Define a nice break command that doesn't care if a line doesn't already
    % exist.
    \def\br{\hspace*{\fill} \\* }
    % Math Jax compatibility definitions
    \def\gt{>}
    \def\lt{<}
    \let\Oldtex\TeX
    \let\Oldlatex\LaTeX
    \renewcommand{\TeX}{\textrm{\Oldtex}}
    \renewcommand{\LaTeX}{\textrm{\Oldlatex}}
    % Document parameters
    % Document title
    \title{{\Huge Bigdata Project Report}}
    % \author{
        % \LARGE{Team \#1}\\
        % \normalsize{Mahmoud Othman Adas} \small{\texttt{SEC:2, BN:19}}\\
        % \normalsize{Yosry Mohamed Yosry} \small{\texttt{SEC:2, BN:33}}\\
        % \normalsize{Ahmad Mahmoud AbdElMen'em Afifi} \small{\texttt{SEC:1, BN:5}}\\
        % \normalsize{Abdulrahman Khalid Hassan} \small{\texttt{SEC:1, BN:30}}
        % \begin{verbatim*}
        %     +---------------------------------+-----+----+
        %     |              Name               | SEC | BN |
        %     +---------------------------------+-----+----+
        %     | Abdulrahman Khalid Hassan       |   1 | 30 |
        %     | Mahmoud Othman Adas             |   2 | 19 |
        %     | Yosry Mohamed Yosry             |   2 | 33 |
        %     | Ahmad Mahmoud AbdElMen'em Afifi |   1 |  5 |
        %     +---------------------------------+-----+----+
        % \end{verbatim*}
    % }
    \author{}

    % Pygments definitions
\makeatletter
\def\PY@reset{\let\PY@it=\relax \let\PY@bf=\relax%
    \let\PY@ul=\relax \let\PY@tc=\relax%
    \let\PY@bc=\relax \let\PY@ff=\relax}
\def\PY@tok#1{\csname PY@tok@#1\endcsname}
\def\PY@toks#1+{\ifx\relax#1\empty\else%
    \PY@tok{#1}\expandafter\PY@toks\fi}
\def\PY@do#1{\PY@bc{\PY@tc{\PY@ul{%
    \PY@it{\PY@bf{\PY@ff{#1}}}}}}}
\def\PY#1#2{\PY@reset\PY@toks#1+\relax+\PY@do{#2}}

\expandafter\def\csname PY@tok@w\endcsname{\def\PY@tc##1{\textcolor[rgb]{0.73,0.73,0.73}{##1}}}
\expandafter\def\csname PY@tok@c\endcsname{\let\PY@it=\textit\def\PY@tc##1{\textcolor[rgb]{0.25,0.50,0.50}{##1}}}
\expandafter\def\csname PY@tok@cp\endcsname{\def\PY@tc##1{\textcolor[rgb]{0.74,0.48,0.00}{##1}}}
\expandafter\def\csname PY@tok@k\endcsname{\let\PY@bf=\textbf\def\PY@tc##1{\textcolor[rgb]{0.00,0.50,0.00}{##1}}}
\expandafter\def\csname PY@tok@kp\endcsname{\def\PY@tc##1{\textcolor[rgb]{0.00,0.50,0.00}{##1}}}
\expandafter\def\csname PY@tok@kt\endcsname{\def\PY@tc##1{\textcolor[rgb]{0.69,0.00,0.25}{##1}}}
\expandafter\def\csname PY@tok@o\endcsname{\def\PY@tc##1{\textcolor[rgb]{0.40,0.40,0.40}{##1}}}
\expandafter\def\csname PY@tok@ow\endcsname{\let\PY@bf=\textbf\def\PY@tc##1{\textcolor[rgb]{0.67,0.13,1.00}{##1}}}
\expandafter\def\csname PY@tok@nb\endcsname{\def\PY@tc##1{\textcolor[rgb]{0.00,0.50,0.00}{##1}}}
\expandafter\def\csname PY@tok@nf\endcsname{\def\PY@tc##1{\textcolor[rgb]{0.00,0.00,1.00}{##1}}}
\expandafter\def\csname PY@tok@nc\endcsname{\let\PY@bf=\textbf\def\PY@tc##1{\textcolor[rgb]{0.00,0.00,1.00}{##1}}}
\expandafter\def\csname PY@tok@nn\endcsname{\let\PY@bf=\textbf\def\PY@tc##1{\textcolor[rgb]{0.00,0.00,1.00}{##1}}}
\expandafter\def\csname PY@tok@ne\endcsname{\let\PY@bf=\textbf\def\PY@tc##1{\textcolor[rgb]{0.82,0.25,0.23}{##1}}}
\expandafter\def\csname PY@tok@nv\endcsname{\def\PY@tc##1{\textcolor[rgb]{0.10,0.09,0.49}{##1}}}
\expandafter\def\csname PY@tok@no\endcsname{\def\PY@tc##1{\textcolor[rgb]{0.53,0.00,0.00}{##1}}}
\expandafter\def\csname PY@tok@nl\endcsname{\def\PY@tc##1{\textcolor[rgb]{0.63,0.63,0.00}{##1}}}
\expandafter\def\csname PY@tok@ni\endcsname{\let\PY@bf=\textbf\def\PY@tc##1{\textcolor[rgb]{0.60,0.60,0.60}{##1}}}
\expandafter\def\csname PY@tok@na\endcsname{\def\PY@tc##1{\textcolor[rgb]{0.49,0.56,0.16}{##1}}}
\expandafter\def\csname PY@tok@nt\endcsname{\let\PY@bf=\textbf\def\PY@tc##1{\textcolor[rgb]{0.00,0.50,0.00}{##1}}}
\expandafter\def\csname PY@tok@nd\endcsname{\def\PY@tc##1{\textcolor[rgb]{0.67,0.13,1.00}{##1}}}
\expandafter\def\csname PY@tok@s\endcsname{\def\PY@tc##1{\textcolor[rgb]{0.73,0.13,0.13}{##1}}}
\expandafter\def\csname PY@tok@sd\endcsname{\let\PY@it=\textit\def\PY@tc##1{\textcolor[rgb]{0.73,0.13,0.13}{##1}}}
\expandafter\def\csname PY@tok@si\endcsname{\let\PY@bf=\textbf\def\PY@tc##1{\textcolor[rgb]{0.73,0.40,0.53}{##1}}}
\expandafter\def\csname PY@tok@se\endcsname{\let\PY@bf=\textbf\def\PY@tc##1{\textcolor[rgb]{0.73,0.40,0.13}{##1}}}
\expandafter\def\csname PY@tok@sr\endcsname{\def\PY@tc##1{\textcolor[rgb]{0.73,0.40,0.53}{##1}}}
\expandafter\def\csname PY@tok@ss\endcsname{\def\PY@tc##1{\textcolor[rgb]{0.10,0.09,0.49}{##1}}}
\expandafter\def\csname PY@tok@sx\endcsname{\def\PY@tc##1{\textcolor[rgb]{0.00,0.50,0.00}{##1}}}
\expandafter\def\csname PY@tok@m\endcsname{\def\PY@tc##1{\textcolor[rgb]{0.40,0.40,0.40}{##1}}}
\expandafter\def\csname PY@tok@gh\endcsname{\let\PY@bf=\textbf\def\PY@tc##1{\textcolor[rgb]{0.00,0.00,0.50}{##1}}}
\expandafter\def\csname PY@tok@gu\endcsname{\let\PY@bf=\textbf\def\PY@tc##1{\textcolor[rgb]{0.50,0.00,0.50}{##1}}}
\expandafter\def\csname PY@tok@gd\endcsname{\def\PY@tc##1{\textcolor[rgb]{0.63,0.00,0.00}{##1}}}
\expandafter\def\csname PY@tok@gi\endcsname{\def\PY@tc##1{\textcolor[rgb]{0.00,0.63,0.00}{##1}}}
\expandafter\def\csname PY@tok@gr\endcsname{\def\PY@tc##1{\textcolor[rgb]{1.00,0.00,0.00}{##1}}}
\expandafter\def\csname PY@tok@ge\endcsname{\let\PY@it=\textit}
\expandafter\def\csname PY@tok@gs\endcsname{\let\PY@bf=\textbf}
\expandafter\def\csname PY@tok@gp\endcsname{\let\PY@bf=\textbf\def\PY@tc##1{\textcolor[rgb]{0.00,0.00,0.50}{##1}}}
\expandafter\def\csname PY@tok@go\endcsname{\def\PY@tc##1{\textcolor[rgb]{0.53,0.53,0.53}{##1}}}
\expandafter\def\csname PY@tok@gt\endcsname{\def\PY@tc##1{\textcolor[rgb]{0.00,0.27,0.87}{##1}}}
\expandafter\def\csname PY@tok@err\endcsname{\def\PY@bc##1{\setlength{\fboxsep}{0pt}\fcolorbox[rgb]{1.00,0.00,0.00}{1,1,1}{\strut ##1}}}
\expandafter\def\csname PY@tok@kc\endcsname{\let\PY@bf=\textbf\def\PY@tc##1{\textcolor[rgb]{0.00,0.50,0.00}{##1}}}
\expandafter\def\csname PY@tok@kd\endcsname{\let\PY@bf=\textbf\def\PY@tc##1{\textcolor[rgb]{0.00,0.50,0.00}{##1}}}
\expandafter\def\csname PY@tok@kn\endcsname{\let\PY@bf=\textbf\def\PY@tc##1{\textcolor[rgb]{0.00,0.50,0.00}{##1}}}
\expandafter\def\csname PY@tok@kr\endcsname{\let\PY@bf=\textbf\def\PY@tc##1{\textcolor[rgb]{0.00,0.50,0.00}{##1}}}
\expandafter\def\csname PY@tok@bp\endcsname{\def\PY@tc##1{\textcolor[rgb]{0.00,0.50,0.00}{##1}}}
\expandafter\def\csname PY@tok@fm\endcsname{\def\PY@tc##1{\textcolor[rgb]{0.00,0.00,1.00}{##1}}}
\expandafter\def\csname PY@tok@vc\endcsname{\def\PY@tc##1{\textcolor[rgb]{0.10,0.09,0.49}{##1}}}
\expandafter\def\csname PY@tok@vg\endcsname{\def\PY@tc##1{\textcolor[rgb]{0.10,0.09,0.49}{##1}}}
\expandafter\def\csname PY@tok@vi\endcsname{\def\PY@tc##1{\textcolor[rgb]{0.10,0.09,0.49}{##1}}}
\expandafter\def\csname PY@tok@vm\endcsname{\def\PY@tc##1{\textcolor[rgb]{0.10,0.09,0.49}{##1}}}
\expandafter\def\csname PY@tok@sa\endcsname{\def\PY@tc##1{\textcolor[rgb]{0.73,0.13,0.13}{##1}}}
\expandafter\def\csname PY@tok@sb\endcsname{\def\PY@tc##1{\textcolor[rgb]{0.73,0.13,0.13}{##1}}}
\expandafter\def\csname PY@tok@sc\endcsname{\def\PY@tc##1{\textcolor[rgb]{0.73,0.13,0.13}{##1}}}
\expandafter\def\csname PY@tok@dl\endcsname{\def\PY@tc##1{\textcolor[rgb]{0.73,0.13,0.13}{##1}}}
\expandafter\def\csname PY@tok@s2\endcsname{\def\PY@tc##1{\textcolor[rgb]{0.73,0.13,0.13}{##1}}}
\expandafter\def\csname PY@tok@sh\endcsname{\def\PY@tc##1{\textcolor[rgb]{0.73,0.13,0.13}{##1}}}
\expandafter\def\csname PY@tok@s1\endcsname{\def\PY@tc##1{\textcolor[rgb]{0.73,0.13,0.13}{##1}}}
\expandafter\def\csname PY@tok@mb\endcsname{\def\PY@tc##1{\textcolor[rgb]{0.40,0.40,0.40}{##1}}}
\expandafter\def\csname PY@tok@mf\endcsname{\def\PY@tc##1{\textcolor[rgb]{0.40,0.40,0.40}{##1}}}
\expandafter\def\csname PY@tok@mh\endcsname{\def\PY@tc##1{\textcolor[rgb]{0.40,0.40,0.40}{##1}}}
\expandafter\def\csname PY@tok@mi\endcsname{\def\PY@tc##1{\textcolor[rgb]{0.40,0.40,0.40}{##1}}}
\expandafter\def\csname PY@tok@il\endcsname{\def\PY@tc##1{\textcolor[rgb]{0.40,0.40,0.40}{##1}}}
\expandafter\def\csname PY@tok@mo\endcsname{\def\PY@tc##1{\textcolor[rgb]{0.40,0.40,0.40}{##1}}}
\expandafter\def\csname PY@tok@ch\endcsname{\let\PY@it=\textit\def\PY@tc##1{\textcolor[rgb]{0.25,0.50,0.50}{##1}}}
\expandafter\def\csname PY@tok@cm\endcsname{\let\PY@it=\textit\def\PY@tc##1{\textcolor[rgb]{0.25,0.50,0.50}{##1}}}
\expandafter\def\csname PY@tok@cpf\endcsname{\let\PY@it=\textit\def\PY@tc##1{\textcolor[rgb]{0.25,0.50,0.50}{##1}}}
\expandafter\def\csname PY@tok@c1\endcsname{\let\PY@it=\textit\def\PY@tc##1{\textcolor[rgb]{0.25,0.50,0.50}{##1}}}
\expandafter\def\csname PY@tok@cs\endcsname{\let\PY@it=\textit\def\PY@tc##1{\textcolor[rgb]{0.25,0.50,0.50}{##1}}}

\def\PYZbs{\char`\\}
\def\PYZus{\char`\_}
\def\PYZob{\char`\{}
\def\PYZcb{\char`\}}
\def\PYZca{\char`\^}
\def\PYZam{\char`\&}
\def\PYZlt{\char`\<}
\def\PYZgt{\char`\>}
\def\PYZsh{\char`\#}
\def\PYZpc{\char`\%}
\def\PYZdl{\char`\$}
\def\PYZhy{\char`\-}
\def\PYZsq{\char`\'}
\def\PYZdq{\char`\"}
\def\PYZti{\char`\~}
% for compatibility with earlier versions
\def\PYZat{@}
\def\PYZlb{[}
\def\PYZrb{]}
\makeatother


    % For linebreaks inside Verbatim environment from package fancyvrb. 
    \makeatletter
        \newbox\Wrappedcontinuationbox 
        \newbox\Wrappedvisiblespacebox 
        \newcommand*\Wrappedvisiblespace {\textcolor{red}{\textvisiblespace}} 
        \newcommand*\Wrappedcontinuationsymbol {\textcolor{red}{\llap{\tiny$\m@th\hookrightarrow$}}} 
        \newcommand*\Wrappedcontinuationindent {3ex } 
        \newcommand*\Wrappedafterbreak {\kern\Wrappedcontinuationindent\copy\Wrappedcontinuationbox} 
        % Take advantage of the already applied Pygments mark-up to insert 
        % potential linebreaks for TeX processing. 
        %        {, <, #, %, $, ' and ": go to next line. 
        %        _, }, ^, &, >, - and ~: stay at end of broken line. 
        % Use of \textquotesingle for straight quote. 
        \newcommand*\Wrappedbreaksatspecials {% 
            \def\PYGZus{\discretionary{\char`\_}{\Wrappedafterbreak}{\char`\_}}% 
            \def\PYGZob{\discretionary{}{\Wrappedafterbreak\char`\{}{\char`\{}}% 
            \def\PYGZcb{\discretionary{\char`\}}{\Wrappedafterbreak}{\char`\}}}% 
            \def\PYGZca{\discretionary{\char`\^}{\Wrappedafterbreak}{\char`\^}}% 
            \def\PYGZam{\discretionary{\char`\&}{\Wrappedafterbreak}{\char`\&}}% 
            \def\PYGZlt{\discretionary{}{\Wrappedafterbreak\char`\<}{\char`\<}}% 
            \def\PYGZgt{\discretionary{\char`\>}{\Wrappedafterbreak}{\char`\>}}% 
            \def\PYGZsh{\discretionary{}{\Wrappedafterbreak\char`\#}{\char`\#}}% 
            \def\PYGZpc{\discretionary{}{\Wrappedafterbreak\char`\%}{\char`\%}}% 
            \def\PYGZdl{\discretionary{}{\Wrappedafterbreak\char`\$}{\char`\$}}% 
            \def\PYGZhy{\discretionary{\char`\-}{\Wrappedafterbreak}{\char`\-}}% 
            \def\PYGZsq{\discretionary{}{\Wrappedafterbreak\textquotesingle}{\textquotesingle}}% 
            \def\PYGZdq{\discretionary{}{\Wrappedafterbreak\char`\"}{\char`\"}}% 
            \def\PYGZti{\discretionary{\char`\~}{\Wrappedafterbreak}{\char`\~}}% 
        } 
        % Some characters . , ; ? ! / are not pygmentized. 
        % This macro makes them "active" and they will insert potential linebreaks 
        \newcommand*\Wrappedbreaksatpunct {% 
            \lccode`\~`\.\lowercase{\def~}{\discretionary{\hbox{\char`\.}}{\Wrappedafterbreak}{\hbox{\char`\.}}}% 
            \lccode`\~`\,\lowercase{\def~}{\discretionary{\hbox{\char`\,}}{\Wrappedafterbreak}{\hbox{\char`\,}}}% 
            \lccode`\~`\;\lowercase{\def~}{\discretionary{\hbox{\char`\;}}{\Wrappedafterbreak}{\hbox{\char`\;}}}% 
            \lccode`\~`\:\lowercase{\def~}{\discretionary{\hbox{\char`\:}}{\Wrappedafterbreak}{\hbox{\char`\:}}}% 
            \lccode`\~`\?\lowercase{\def~}{\discretionary{\hbox{\char`\?}}{\Wrappedafterbreak}{\hbox{\char`\?}}}% 
            \lccode`\~`\!\lowercase{\def~}{\discretionary{\hbox{\char`\!}}{\Wrappedafterbreak}{\hbox{\char`\!}}}% 
            \lccode`\~`\/\lowercase{\def~}{\discretionary{\hbox{\char`\/}}{\Wrappedafterbreak}{\hbox{\char`\/}}}% 
            \catcode`\.\active
            \catcode`\,\active 
            \catcode`\;\active
            \catcode`\:\active
            \catcode`\?\active
            \catcode`\!\active
            \catcode`\/\active 
            \lccode`\~`\~ 	
        }
    \makeatother

    \let\OriginalVerbatim=\Verbatim
    \makeatletter
    \renewcommand{\Verbatim}[1][1]{%
        %\parskip\z@skip
        \sbox\Wrappedcontinuationbox {\Wrappedcontinuationsymbol}%
        \sbox\Wrappedvisiblespacebox {\FV@SetupFont\Wrappedvisiblespace}%
        \def\FancyVerbFormatLine ##1{\hsize\linewidth
            \vtop{\raggedright\hyphenpenalty\z@\exhyphenpenalty\z@
                \doublehyphendemerits\z@\finalhyphendemerits\z@
                \strut ##1\strut}%
        }%
        % If the linebreak is at a space, the latter will be displayed as visible
        % space at end of first line, and a continuation symbol starts next line.
        % Stretch/shrink are however usually zero for typewriter font.
        \def\FV@Space {%
            \nobreak\hskip\z@ plus\fontdimen3\font minus\fontdimen4\font
            \discretionary{\copy\Wrappedvisiblespacebox}{\Wrappedafterbreak}
            {\kern\fontdimen2\font}%
        }%
        
        % Allow breaks at special characters using \PYG... macros.
        \Wrappedbreaksatspecials
        % Breaks at punctuation characters . , ; ? ! and / need catcode=\active 	
        \OriginalVerbatim[#1,codes*=\Wrappedbreaksatpunct]%
    }
    \makeatother

    % Exact colors from NB
    \definecolor{incolor}{HTML}{303F9F}
    \definecolor{outcolor}{HTML}{D84315}
    \definecolor{cellborder}{HTML}{CFCFCF}
    \definecolor{cellbackground}{HTML}{F7F7F7}
    
    % prompt
    \makeatletter
    \newcommand{\boxspacing}{\kern\kvtcb@left@rule\kern\kvtcb@boxsep}
    \makeatother
    \newcommand{\prompt}[4]{
        {\ttfamily\llap{{\color{#2}[#3]:\hspace{3pt}#4}}\vspace{-\baselineskip}}
    }
    

    
    % Prevent overflowing lines due to hard-to-break entities
    \sloppy 
    % Setup hyperref package
    \hypersetup{
      breaklinks=true,  % so long urls are correctly broken across lines
      colorlinks=true,
      urlcolor=urlcolor,
      linkcolor=linkcolor,
      citecolor=citecolor,
      }
    % Slightly bigger margins than the latex defaults
    
    \geometry{verbose,tmargin=1in,bmargin=1in,lmargin=1in,rmargin=1in}
    
    

\begin{document}
    
    \maketitle
    
    
    {\Large 
    \begin{center}
        \begin{verbatim}
                             Team #1
        +---------------------------------+-----+----+
        |              Name               | SEC | BN |
        +---------------------------------+-----+----+
        | Abdulrahman Khalid Hassan       |   1 | 30 |
        | Mahmoud Othman Adas             |   2 | 19 |
        | Yosry Mohamed Yosry             |   2 | 33 |
        | Ahmad Mahmoud AbdElMen'em Afifi |   1 |  5 |
        +---------------------------------+-----+----+
        \end{verbatim}
    \end{center}
    }
    
    \section{Introduction}
    

    LendingClub is a US peer-to-peer lending company, headquartered in San
Francisco, California. It was the first peer-to-peer lender to register
its offerings as securities with the Securities and Exchange Commission,
and to offer loan trading on a secondary market. LendingClub is the
world's largest peer-to-peer lending platform, Given historical data on
loans given out with information on whether or not the borrower
defaulted (charge-off), we can build a model that can predict if a
borrower will pay back their loan. This way in the future when we get a
new potential customer, we can assess if they are likely to pay back the
loan.

Objectives of this notebook is: - To show step-by-step how to visualize
the dataset. - Data cleaning and preprocessing. - Assess whether or not
a new customer is likely to pay back the loan.

    \# Univariant Visualization

    \hypertarget{loan-status-distribution}{%
\subsection{Loan Status Distribution}\label{loan-status-distribution}}

    \begin{Verbatim}[commandchars=\\\{\}]
+--------------------+-------+
|         loan\_status|  count|
+--------------------+-------+
|          Fully Paid|1076751|
|             Default|     40|
|                null|     33|
|     In Grace Period|   8436|
|Does not meet the{\ldots}|   1988|
|         Charged Off| 268558|
|            Oct-2015|      1|
|  Late (31-120 days)|  21467|
|             Current| 878317|
|Does not meet the{\ldots}|    761|
|   Late (16-30 days)|   4349|
+--------------------+-------+

    \end{Verbatim}

    \begin{center}
    \adjustimage{max size={0.9\linewidth}{0.9\paperheight}}{report_files/report_5_0.png}
    \end{center}
    { \hspace*{\fill} \\}
    
    We filter the loan status to be only fully paid and charged off

    \begin{Verbatim}[commandchars=\\\{\}]
+-----------+-------+
|loan\_status|  count|
+-----------+-------+
| Fully Paid|1076751|
|Charged Off| 268558|
+-----------+-------+

    \end{Verbatim}

    \begin{center}
    \adjustimage{max size={0.9\linewidth}{0.9\paperheight}}{report_files/report_8_0.png}
    \end{center}
    { \hspace*{\fill} \\}
    
    \begin{center}
    \adjustimage{max size={0.9\linewidth}{0.9\paperheight}}{report_files/report_9_0.png}
    \end{center}
    { \hspace*{\fill} \\}
    
    \hypertarget{handling-null-values}{%
\subsection{Handling Null Values}\label{handling-null-values}}

\begin{itemize}
\tightlist
\item
  Get columns which has the most number of null values and sort them
\end{itemize}

    \begin{itemize}
\tightlist
\item
  Visualize the most 50 columns with null values
\end{itemize}

    \begin{center}
    \adjustimage{max size={0.9\linewidth}{0.9\paperheight}}{report_files/report_12_0.png}
    \end{center}
    { \hspace*{\fill} \\}
    
    The first most 50 columns with highest numbers of null values (Almost
all the values are null as number of rows are 1345309 initially) So we
have to drop them all as deletion of the rows means the deletion of most
of the data and I can't replace it mean value as most of the values are
null it won't be the best option, and also if these columns are
important they would be filled, so We will drop them

    \begin{itemize}
\tightlist
\item
  Next, We visualize the next most 50 columns have null values count
\end{itemize}

    \begin{center}
    \adjustimage{max size={0.9\linewidth}{0.9\paperheight}}{report_files/report_15_0.png}
    \end{center}
    { \hspace*{\fill} \\}
    
    \begin{itemize}
\tightlist
\item
  From the previous graph the first 95 columns has lots of nulls so I
  will drop them
\item
  Next, We will drop the rows which has null values they will have a
  small number of rows
\item
  Now the Number Of Columns = 56, Number Of Rows = 1340812
\end{itemize}

    \hypertarget{loan-amount-distribution}{%
\subsection{Loan Amount Distribution}\label{loan-amount-distribution}}

    \begin{center}
    \adjustimage{max size={0.9\linewidth}{0.9\paperheight}}{report_files/report_18_0.png}
    \end{center}
    { \hspace*{\fill} \\}
    
    \begin{itemize}
\tightlist
\item
  Most of loan amount 10,000 to 11,000 usd
\end{itemize}

    \begin{center}
    \adjustimage{max size={0.9\linewidth}{0.9\paperheight}}{report_files/report_20_0.png}
    \end{center}
    { \hspace*{\fill} \\}
    
    \begin{center}
    \adjustimage{max size={0.9\linewidth}{0.9\paperheight}}{report_files/report_21_0.png}
    \end{center}
    { \hspace*{\fill} \\}
    
    From the above graphs, We can deduce that we have duplicates information
for example, Drop the followin columns as they are constant columns and
doesn't contribute to our prediction of loan\_status

    \# Bivariant Visualization

    \begin{center}
    \adjustimage{max size={0.9\linewidth}{0.9\paperheight}}{report_files/report_25_0.png}
    \end{center}
    { \hspace*{\fill} \\}
    
    \begin{quote}
From the previous Correlation Matrix (policy\_code, out\_prncp,
out\_prncp\_inv) don't have any correlation with any other columns so
drop them
\end{quote}

    \begin{quote}
Next, Check for redundant information columns: check pairs of features
which has correlation value above 0.8
\end{quote}

            \begin{tcolorbox}[breakable, size=fbox, boxrule=.5pt, pad at break*=1mm, opacityfill=0]
\prompt{Out}{outcolor}{42}{\boxspacing}
\begin{Verbatim}[commandchars=\\\{\}]
                   feature1              feature2      corr
1           fico\_range\_high        fico\_range\_low  1.000000
2                 loan\_amnt           funded\_amnt  0.999567
3           total\_pymnt\_inv           total\_pymnt  0.999548
4           funded\_amnt\_inv           funded\_amnt  0.999447
5           funded\_amnt\_inv             loan\_amnt  0.998929
6   collection\_recovery\_fee            recoveries  0.972815
7           total\_rec\_prncp           total\_pymnt  0.967105
8           total\_rec\_prncp       total\_pymnt\_inv  0.966732
9               funded\_amnt           installment  0.954036
10              installment       funded\_amnt\_inv  0.953455
11              installment             loan\_amnt  0.953388
12          total\_pymnt\_inv       funded\_amnt\_inv  0.857143
13              total\_pymnt           funded\_amnt  0.856896
14              funded\_amnt       total\_pymnt\_inv  0.856674
15              total\_pymnt             loan\_amnt  0.856653
16              total\_pymnt       funded\_amnt\_inv  0.856447
17                loan\_amnt       total\_pymnt\_inv  0.856354
18      last\_fico\_range\_low  last\_fico\_range\_high  0.829738
19              installment           total\_pymnt  0.818284
20          total\_pymnt\_inv           installment  0.818049
\end{Verbatim}
\end{tcolorbox}
        
    It seems that the data have many duplicated information represented by
multiple columns, so let's drop these duplicated columns

\begin{itemize}
\tightlist
\item
  fico\_range\_low = (fico\_range\_high)
\item
  funded\_amnt = funded\_amnt\_inv = installment = total\_pymnt\_inv =
  total\_rec\_prncp = total\_pymnt = (loan\_amnt)
\item
  collection\_recovery\_fee = (recoveries)
\item
  last\_fico\_range\_low = (last\_fico\_range\_high)
\end{itemize}

\begin{quote}
Drop them all except the columns between brackets () to avoid
information redundancy
\end{quote}

    \begin{center}
    \adjustimage{max size={0.9\linewidth}{0.9\paperheight}}{report_files/report_31_0.png}
    \end{center}
    { \hspace*{\fill} \\}
    
    \begin{center}
    \adjustimage{max size={0.9\linewidth}{0.9\paperheight}}{report_files/report_32_0.png}
    \end{center}
    { \hspace*{\fill} \\}
    
    \begin{center}
    \adjustimage{max size={0.9\linewidth}{0.9\paperheight}}{report_files/report_33_0.png}
    \end{center}
    { \hspace*{\fill} \\}
    
    \begin{center}
    \adjustimage{max size={0.9\linewidth}{0.9\paperheight}}{report_files/report_34_0.png}
    \end{center}
    { \hspace*{\fill} \\}
    
    \begin{quote}
After dropping these columns
\end{quote}

    \begin{Verbatim}[commandchars=\\\{\}]
(1340812, 41)
    \end{Verbatim}

    \# Preprocessing

    \hypertarget{loan_status-column}{%
\subsubsection{loan\_status Column}\label{loan_status-column}}

    Map loan\_status to new column called loan\_is\_paid has 1 and 0 values
only, 1 for Fully Paid and 0 for Charged Off, and then drop the old
loan\_status column

    \begin{Verbatim}[commandchars=\\\{\}]
+------------+-------+
|loan\_is\_paid|  count|
+------------+-------+
|           1|1074961|
|           0| 265851|
+------------+-------+

    \end{Verbatim}

    New Correlation matrix:

    \begin{center}
    \adjustimage{max size={0.9\linewidth}{0.9\paperheight}}{report_files/report_44_0.png}
    \end{center}
    { \hspace*{\fill} \\}
    
    \begin{quote}
Correlation between loan\_is\_paid and all other numeric columns
\end{quote}

    \begin{center}
    \adjustimage{max size={0.9\linewidth}{0.9\paperheight}}{report_files/report_46_0.png}
    \end{center}
    { \hspace*{\fill} \\}
    
    \hypertarget{term-column}{%
\subsubsection{term Column}\label{term-column}}

    map term column to new term\_months columns with mapped values from ' 36
months' to 36 and from ' 60 months' to 60, and then drop the old term
column

            \begin{tcolorbox}[breakable, size=fbox, boxrule=.5pt, pad at break*=1mm, opacityfill=0]
\prompt{Out}{outcolor}{44}{\boxspacing}
\begin{Verbatim}[commandchars=\\\{\}]
[' 36 months', ' 60 months']
\end{Verbatim}
\end{tcolorbox}
        
    \begin{Verbatim}[commandchars=\\\{\}]
+-----------+
|term\_months|
+-----------+
|         60|
|         36|
+-----------+

    \end{Verbatim}

    \hypertarget{home_ownership-column}{%
\subsubsection{home\_ownership Column}\label{home_ownership-column}}

    We can merge NONE with ANY in one category

    \begin{Verbatim}[commandchars=\\\{\}]
+--------------+------+
|home\_ownership| count|
+--------------+------+
|           OWN|144179|
|          RENT|532381|
|      MORTGAGE|663782|
|           ANY|   328|
|         OTHER|   142|
+--------------+------+

    \end{Verbatim}

    \hypertarget{grade-and-sub_grade-columns}{%
\subsubsection{grade And sub\_grade
Columns}\label{grade-and-sub_grade-columns}}

    \begin{center}
    \adjustimage{max size={0.9\linewidth}{0.9\paperheight}}{report_files/report_60_0.png}
    \end{center}
    { \hspace*{\fill} \\}
    
    \begin{center}
    \adjustimage{max size={0.9\linewidth}{0.9\paperheight}}{report_files/report_61_0.png}
    \end{center}
    { \hspace*{\fill} \\}
    
    From the above graphs and analysis we can deduce that:

\begin{itemize}
\item
  grade is part of sub\_grade, so let's drop it
\item
  Date columns: issue\_d, last\_pymnt\_d, last\_credit\_pull\_d are not
  important to the analysis
\item
  earliest\_cr\_line which is the month when reported credit line was
  opened is not important to the analysis
\item
  url for LC page with listing data is not important to the analysis
\item
  addresses: zip\_code, addr\_state are not important to the analysis
\end{itemize}

    \hypertarget{handle-categorical-features}{%
\subsection{Handle Categorical
Features}\label{handle-categorical-features}}

    \hypertarget{spark-pipeline}{%
\subsubsection{Spark Pipeline}\label{spark-pipeline}}

\begin{enumerate}
\def\labelenumi{\arabic{enumi}.}
\tightlist
\item
  Categorical columns to string indexer to change categories to numbers
\item
  OneHotEncode these new numbers to (\#num of column - 1) new column
  every column value with has 1 whenever this category happens to be in
  this row
\item
  Assemble all the features the onehotencoded and the numeric columns in
  one vector columns used as feature column
\item
  Get scaled\_feature column by scaling feature column using MinMaxcaler
\end{enumerate}

    Select now the two important columns used in building the model: -
scaled\_features - loan\_is\_paid

    Schema:

    \begin{Verbatim}[commandchars=\\\{\}]
root
 |-- scaled\_features: vector (nullable = true)
 |-- loan\_is\_paid: integer (nullable = true)

    \end{Verbatim}

    First Row Of the dataframe:

    \begin{Verbatim}[commandchars=\\\{\}]
+--------------------+------------+
|     scaled\_features|loan\_is\_paid|
+--------------------+------------+
|(81,[1,2,4,5,6,8,{\ldots}|           1|
+--------------------+------------+
only showing top 1 row

    \end{Verbatim}

    From previous table the number of scaled\_features are 81 feature

    \# Deeplearning Model

    \begin{enumerate}
\def\labelenumi{\arabic{enumi}.}
\tightlist
\item
  Build the following deeplearning model with this output shapes
  consists of 6 fully connected layer with relu function as the
  activation function and the final activation function is a sigmoid
\item
  Compile the model with adam optimizer with accuracy metrics and
  binary\_crossentropy loss function
\end{enumerate}

    \begin{Verbatim}[commandchars=\\\{\}]
Model: "sequential\_1"
\_\_\_\_\_\_\_\_\_\_\_\_\_\_\_\_\_\_\_\_\_\_\_\_\_\_\_\_\_\_\_\_\_\_\_\_\_\_\_\_\_\_\_\_\_\_\_\_\_\_\_\_\_\_\_\_\_\_\_\_\_\_\_\_\_
Layer (type)                 Output Shape              Param \#
=================================================================
dense\_6 (Dense)              (None, 78)                6396
\_\_\_\_\_\_\_\_\_\_\_\_\_\_\_\_\_\_\_\_\_\_\_\_\_\_\_\_\_\_\_\_\_\_\_\_\_\_\_\_\_\_\_\_\_\_\_\_\_\_\_\_\_\_\_\_\_\_\_\_\_\_\_\_\_
dense\_7 (Dense)              (None, 39)                3081
\_\_\_\_\_\_\_\_\_\_\_\_\_\_\_\_\_\_\_\_\_\_\_\_\_\_\_\_\_\_\_\_\_\_\_\_\_\_\_\_\_\_\_\_\_\_\_\_\_\_\_\_\_\_\_\_\_\_\_\_\_\_\_\_\_
dense\_8 (Dense)              (None, 19)                760
\_\_\_\_\_\_\_\_\_\_\_\_\_\_\_\_\_\_\_\_\_\_\_\_\_\_\_\_\_\_\_\_\_\_\_\_\_\_\_\_\_\_\_\_\_\_\_\_\_\_\_\_\_\_\_\_\_\_\_\_\_\_\_\_\_
dense\_9 (Dense)              (None, 8)                 160
\_\_\_\_\_\_\_\_\_\_\_\_\_\_\_\_\_\_\_\_\_\_\_\_\_\_\_\_\_\_\_\_\_\_\_\_\_\_\_\_\_\_\_\_\_\_\_\_\_\_\_\_\_\_\_\_\_\_\_\_\_\_\_\_\_
dense\_10 (Dense)             (None, 4)                 36
\_\_\_\_\_\_\_\_\_\_\_\_\_\_\_\_\_\_\_\_\_\_\_\_\_\_\_\_\_\_\_\_\_\_\_\_\_\_\_\_\_\_\_\_\_\_\_\_\_\_\_\_\_\_\_\_\_\_\_\_\_\_\_\_\_
dense\_11 (Dense)             (None, 1)                 5
=================================================================
Total params: 10,438
Trainable params: 10,438
Non-trainable params: 0
\_\_\_\_\_\_\_\_\_\_\_\_\_\_\_\_\_\_\_\_\_\_\_\_\_\_\_\_\_\_\_\_\_\_\_\_\_\_\_\_\_\_\_\_\_\_\_\_\_\_\_\_\_\_\_\_\_\_\_\_\_\_\_\_\_
    \end{Verbatim}

    \begin{enumerate}
\def\labelenumi{\arabic{enumi}.}
\setcounter{enumi}{2}
\tightlist
\item
  Transform to the dataframe to pandas dataframe before training
\end{enumerate}

    \begin{enumerate}
\def\labelenumi{\arabic{enumi}.}
\setcounter{enumi}{3}
\tightlist
\item
  Split data 80\% training set and 20\% testing set
\end{enumerate}

    \begin{enumerate}
\def\labelenumi{\arabic{enumi}.}
\setcounter{enumi}{4}
\tightlist
\item
  Finally, Train the model
\end{enumerate}

    Loss Graph:

            \begin{tcolorbox}[breakable, size=fbox, boxrule=.5pt, pad at break*=1mm, opacityfill=0]
\prompt{Out}{outcolor}{71}{\boxspacing}
\begin{Verbatim}[commandchars=\\\{\}]
<AxesSubplot:>
\end{Verbatim}
\end{tcolorbox}
        
    \begin{center}
    \adjustimage{max size={0.9\linewidth}{0.9\paperheight}}{report_files/report_88_1.png}
    \end{center}
    { \hspace*{\fill} \\}
    
    Accuracy Graph:

            \begin{tcolorbox}[breakable, size=fbox, boxrule=.5pt, pad at break*=1mm, opacityfill=0]
\prompt{Out}{outcolor}{72}{\boxspacing}
\begin{Verbatim}[commandchars=\\\{\}]
<AxesSubplot:>
\end{Verbatim}
\end{tcolorbox}
        
    \begin{center}
    \adjustimage{max size={0.9\linewidth}{0.9\paperheight}}{report_files/report_90_1.png}
    \end{center}
    { \hspace*{\fill} \\}
    
    \textbf{Accuracy = 98.94\%}

    \begin{enumerate}
\def\labelenumi{\arabic{enumi}.}
\setcounter{enumi}{5}
\tightlist
\item
  Saving the model \ldots{}
\end{enumerate}

    \begin{Verbatim}[commandchars=\\\{\}]
INFO:tensorflow:Assets written to: loan\_prediction\_model/assets
    \end{Verbatim}

    \# Gradient Boosting Tree (spark)

    \begin{enumerate}
\def\labelenumi{\arabic{enumi}.}
\tightlist
\item
  Using the map reduce machine learning models of pyspark and spark
  dataframe we can build model using HDFS
\end{enumerate}

    \begin{enumerate}
\def\labelenumi{\arabic{enumi}.}
\setcounter{enumi}{1}
\tightlist
\item
  Split data 80\% training set and 20\% testing set
\item
  Train the Gradient Boosting Tree Classifier using the training data
\end{enumerate}

    \begin{enumerate}
\def\labelenumi{\arabic{enumi}.}
\setcounter{enumi}{3}
\tightlist
\item
  Finally, Saving the model \ldots{}
\end{enumerate}

    Some Predictions:

    \begin{Verbatim}[commandchars=\\\{\}]
+--------------------+------------+----------+--------------------+
|     scaled\_features|loan\_is\_paid|prediction|         probability|
+--------------------+------------+----------+--------------------+
|(81,[0,1,2,3,4,5,{\ldots}|           1|       1.0|[0.05270399993585{\ldots}|
|(81,[0,1,2,3,4,5,{\ldots}|           1|       1.0|[0.05443185145658{\ldots}|
|(81,[0,1,2,3,4,5,{\ldots}|           0|       0.0|[0.95635347857271{\ldots}|
|(81,[0,1,2,3,4,5,{\ldots}|           0|       0.0|[0.95635347857271{\ldots}|
|(81,[0,1,2,3,4,5,{\ldots}|           0|       0.0|[0.95635347857271{\ldots}|
|(81,[0,1,2,3,4,5,{\ldots}|           0|       0.0|[0.95635347857271{\ldots}|
|(81,[0,1,2,3,4,5,{\ldots}|           0|       0.0|[0.95635347857271{\ldots}|
|(81,[0,1,2,3,4,5,{\ldots}|           1|       1.0|[0.04368680010337{\ldots}|
|(81,[0,1,2,3,4,5,{\ldots}|           1|       1.0|[0.08128007245037{\ldots}|
|(81,[0,1,2,3,4,5,{\ldots}|           0|       1.0|[0.27032648846438{\ldots}|
+--------------------+------------+----------+--------------------+
only showing top 10 rows

    \end{Verbatim}

    \begin{Verbatim}[commandchars=\\\{\}]
Test Area Under ROC: 0.9524807005159022
    \end{Verbatim}

    \begin{Verbatim}[commandchars=\\\{\}]
Test f1 score:  0.9744686686161473
    \end{Verbatim}

    \begin{Verbatim}[commandchars=\\\{\}]
Test accuracy:  0.9746691093995363
    \end{Verbatim}

    \begin{Verbatim}[commandchars=\\\{\}]
+------------+----------+------+
|loan\_is\_paid|prediction| count|
+------------+----------+------+
|           1|       0.0|  2304|
|           0|       0.0| 48570|
|           1|       1.0|212115|
|           0|       1.0|  4471|
+------------+----------+------+

    \end{Verbatim}

    Previous table has False Positive, False Negative, True Positive and
True Negative values

    \textbf{Accuracy = 97.46\%}

    \section{Conclusion}

    \begin{itemize}
\tightlist
\item
  We visualized the dataset using univarient and multivarient
  visualization and deduced some information from it
\item
  We've done some Data cleaning and preprocessing to prepare the data to
  build our models
\item
  We using HDFS for both data visualizing, data preprocessing and
  building gradient boosing tree classifier with accuracy 97.46\%
\item
  We also made a deeplearning model with accuracy 98.94\%
\item
  Both models assess whether or not a new customer is likely to pay back
  the loan, so this analysis is very important for decision making.
\end{itemize}


    % Add a bibliography block to the postdoc
    
    
    
\end{document}
